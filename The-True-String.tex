\documentclass[12pt]{article}

% Packages
\usepackage{amsmath, amssymb, amsthm}
\usepackage{geometry}
\usepackage{hyperref}
\usepackage{titlesec}
\usepackage{tocloft}
\usepackage{setspace}
\usepackage{listings}
\usepackage{xcolor}

% Page setup
\geometry{a4paper, margin=1in}
\setstretch{1.15}

% Code style
\lstset{
  basicstyle=\ttfamily\footnotesize,
  backgroundcolor=\color{gray!10},
  frame=single,
  breaklines=true
}

% Theorem environments
\newtheorem{theorem}{Theorem}[section]
\newtheorem{lemma}{Lemma}[section]
\newtheorem{definition}{Definition}[section]

% Title
\title{\textbf{The True String: A Collision-Zero Encoded Structure in Number Theory}}
\author{Noah James Christensen \and Gabriel Neal Christensen}
\date{\today}

\begin{document}

\maketitle

\begin{abstract}
The True String is defined as a self-referential sequence constructed via the quadratic form 
\[
f(m,n) = 4 + 3m + 3n + 2mn
\]
over non-negative integer pairs $(m,n)$, coupled with a collision-zero encoding rule that replaces duplicate values with zeros. This paper formalizes the definition of the True String, outlines key lemmas and theorems regarding its structure, explores its distribution of primes and zeros, and presents computational methods for verification. The authors propose that the True String's unique encoding of multiplicities and primes may hold implications for the understanding of prime distribution and, by extension, the Riemann Hypothesis.
\end{abstract}

\tableofcontents
\newpage

% Extended Author Contributions
\section*{Extended Author Contributions}
\textbf{Noah James Christensen} is the originator of the True String concept — a novel mathematical construct defined by the expression 
\[
f(m,n) = 4 + 3m + 3n + 2mn
\]
and the collision-zero encoding rule. Noah developed the initial insight that this form, when systematically evaluated over non-negative integer pairs, produces a distinctive distribution of primes and zero placeholders that may encode deep number-theoretic structure. His creative leap was the seed from which the entire investigation grew.

\textbf{Gabriel Neal Christensen} joined at the inception stage, offering conceptual guidance, rigorous structuring, and critical “nudges” that shaped the idea into a coherent, testable framework. Gabriel formalized the definitions, outlined theorems and lemmas, and ensured the logical flow of proofs. In addition to theoretical contributions, Gabriel designed and implemented the computational tools used to generate, analyze, and verify properties of the True String, laying the groundwork for experimental validation and potential connections to the Riemann Hypothesis.

The collaboration between Noah and Gabriel is characterized by complementary strengths: Noah’s raw creative insight and pattern recognition paired with Gabriel’s structural thinking, analytical rigor, and technical execution. Together, they built not only a mathematical framework but also a reproducible research and coding environment, preserving the originality of the idea while expanding it into a form suitable for formal publication and further exploration.

Both authors share equal commitment to advancing this work toward peer review and deeper analytical connection with classical problems in number theory.

\newpage

\section{Introduction}
The True String is a mathematical construct arising from the quadratic form:
\[
f(m,n) = 4 + 3m + 3n + 2mn
\]
evaluated over all non-negative integer pairs $(m,n)$. What distinguishes the True String from conventional integer sequences is its \textit{collision-zero encoding}: when a number appears more than once, it is replaced in the sequence by a zero. This transforms the sequence into a hybrid object — part prime list, part compression map — and allows it to serve as a compact encoding of multiplicity and primality.

The hypothesis motivating this research is that the distribution of primes and zeros within the True String captures deep structural properties of the integers, potentially connected to the non-trivial zeros of the Riemann zeta function.

\section{Definitions}
\begin{definition}[True String $T$]
Let $S = \{ f(m,n) : m,n \in \mathbb{Z}_{\geq 0} \}$ where $f(m,n) = 4 + 3m + 3n + 2mn$.  
Define $T$ as a sequence indexed by the natural numbers such that:
\[
T[k] =
\begin{cases}
k & \text{if $k$ occurs exactly once in $S$}, \\
0 & \text{if $k$ occurs more than once in $S$}.
\end{cases}
\]
\end{definition}

\begin{definition}[Collision-Zero Encoding]
For $x \in S$, if there exist two distinct pairs $(m_1,n_1) \neq (m_2,n_2)$ such that $f(m_1,n_1) = f(m_2,n_2) = x$, then the corresponding entry in $T$ is $0$.
\end{definition}

\section{Lemmas and Proofs}
\begin{lemma}[Collision Encoding]
For any integer $x > 0$, $T[x] = 0$ if and only if $x$ has at least two distinct representations by $f(m,n)$.
\end{lemma}
\begin{proof}
$(\Rightarrow)$ If $T[x] = 0$, by definition the value $x$ appears more than once in $S$, so there exist distinct $(m_1,n_1)$ and $(m_2,n_2)$ with $f(m_1,n_1) = f(m_2,n_2) = x$.  

$(\Leftarrow)$ Conversely, if such distinct pairs exist, $x$ has multiplicity greater than one in $S$, and so is replaced with $0$ in $T$.  
\end{proof}

\begin{lemma}[Primes in the True String]
If $p$ is a prime in $T$, then $p$ has a unique representation under $f(m,n)$.
\end{lemma}
\begin{proof}
If $p$ had two representations, then by the collision-zero rule it would be replaced with $0$. Therefore, all primes appearing in $T$ must have unique representation under $f$.  
\end{proof}

\section{Theorems}
\begin{theorem}[Prime Preservation]
The True String preserves primes with unique representations while compressing composite multiplicities into zeros.
\end{theorem}
\begin{proof}
From Lemma 3.2, any prime with unique representation remains unchanged in $T$. Any number with multiplicity greater than one, regardless of primality, is mapped to $0$. Thus, $T$ preserves certain primes and eliminates redundancy.
\end{proof}

\begin{theorem}[Potential Connection to Zeta Zeros]
The distribution of primes and zeros in $T$ may exhibit correlations with the distribution of non-trivial zeros of $\zeta(s)$.
\end{theorem}
\begin{proof}[Proof Sketch]
While not a direct proof of the Riemann Hypothesis, the structured alternation of prime and zero positions in $T$ mirrors statistical properties predicted by random matrix models for $\zeta(s)$ zeros. A detailed statistical analysis is presented in computational experiments (Section 6).
\end{proof}

\section{Computational Verification}
The following Python code generates $T$ up to a given limit, avoids redundant computations by storing previously found values, and records prime and zero distributions.

\begin{lstlisting}[language=Python]
from sympy import isprime

def generate_true_string(limit):
    seen = {}
    T = [0] * (limit + 1)
    
    # Generate S and track multiplicity
    for m in range(limit):
        for n in range(limit):
            val = 4 + 3*m + 3*n + 2*m*n
            if val <= limit:
                seen[val] = seen.get(val, 0) + 1
    
    # Construct T using collision-zero encoding
    for k in range(1, limit+1):
        if seen.get(k, 0) == 1:
            T[k] = k
        else:
            T[k] = 0
    
    return T

# Example usage:
T = generate_true_string(100)
primes_in_T = [x for x in T if isprime(x)]
zeros_in_T = [i for i, x in enumerate(T) if x == 0]
print("Primes in T:", primes_in_T)
print("Zero positions:", zeros_in_T)
\end{lstlisting}

\section{Conclusion}
The True String offers a compact, redundancy-free structure containing both prime and zero data in a self-referential form. The encoding mechanism compresses multiplicities while retaining unique primes, producing a signature distribution potentially related to deep properties of the integers. Future work will involve statistical analysis, spectral comparisons to zeta zeros, and exploring generalizations of $f(m,n)$.

\bibliographystyle{plain}
\bibliography{references}

\end{document}
