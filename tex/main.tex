% !TEX TS-program = pdflatex
\documentclass[11pt]{article}
\usepackage[a4paper,margin=1in]{geometry}
\usepackage{amsmath,amssymb,amsthm}
\usepackage{hyperref}
\usepackage{microtype}
\usepackage{mathtools}
\usepackage{enumitem}
\hypersetup{colorlinks=true,linkcolor=blue,citecolor=blue,urlcolor=blue}

\newtheorem{theorem}{Theorem}
\newtheorem{lemma}{Lemma}
\newtheorem{claim}{Claim}
\newtheorem{corollary}{Corollary}
\theoremstyle{definition}
\newtheorem{definition}{Definition}
\theoremstyle{remark}
\newtheorem{remark}{Remark}

\title{The True String: Combinatorial Structure of Odd Primes and Spectral Connections}
\author{Noah Christensen \and Gabriel Neal Christensen}
\date{\today}

\begin{document}
\maketitle
\tableofcontents

\begin{abstract}
We study the odd-prime indicator sequence defined on indices of odd integers, present a progression-based characterization of odd composites, and explore spectral transforms of the resulting indicator. We organize definitions, statements, and verifiable computations to facilitate independent checking. Empirical sections include code-driven tests and spectral numerics. Formal proof details are outlined with assumptions and dependencies made explicit.
\end{abstract}

\section{Introduction}
Let \(o_n = 2n+1\) denote the \(n\)-th odd integer for \(n \in \mathbb{N}_0\). Define the indicator sequence \(T[n]\) by
\[
T[n] = \begin{cases}1 & \text{if } o_n \text{ is prime},\\ 0 & \text{otherwise.}\end{cases}
\]
We develop a progression-based description for odd composites, an exact odd-only sieve computing \(T[0..N]\), and an experimental spectral analysis of \(T\).

\paragraph{Contributions.} (i) A cleaned, checkable formulation of composite-generating progressions; (ii) Reproducible code for exact computation of \(T\); (iii) Statistical and spectral experiments with transparent limitations.

\section{Definitions}
\begin{definition}[Composite-generating progressions]
For an odd prime \(p\ge 3\) and \(m\in\mathbb{N}_0\), define
\[
 C_p(m) = 3p + 2pm, \quad n_p(m) = \frac{3p-1}{2} + pm.
\]
\end{definition}
The map \(n \mapsto o_n=2n+1\) identifies indices with odd integers.

\begin{lemma}[Complete cover of odd composites]\label{lem:cover}
Every odd composite \(k\ge 9\) has the form \(k = C_p(m)\) for some odd prime \(p\ge 3\) and \(m\in\mathbb{N}_0\). Equivalently, its index \(n\) satisfies \(n = n_p(m)\).
\end{lemma}
\begin{proof}
Let \(p\) be the smallest prime divisor of odd composite \(k\). Then \(k=pr\) with \(r\ge p\) odd. Write \(r=2m+3\) for some \(m\in\mathbb{N}_0\), yielding \(k=3p+2pm=C_p(m)\). The index identity follows since \(k=2n+1\Rightarrow n=(k-1)/2\).
\end{proof}

\section{Exact computation of T}
We provide two equivalent constructions of \(T\) that are implemented in the accompanying code.
\begin{itemize}[leftmargin=*]
  \item Odd-only sieve up to \(2N+1\), producing exact \(T[0..N]\).
  \item Progression marking: for each odd prime \(p\ge3\), mark indices \(n=n_p(m)\) as composite.
\end{itemize}
\begin{theorem}[Equivalence up to a bound]\label{thm:equiv}
For any \(N\ge 0\), the odd-only sieve and progression-marking methods compute identical arrays \(T[0..N]\).
\end{theorem}
\noindent This is verified by code for large \(N\) and is provable by standard properties of sieves and least prime divisors.

\section{Prime counting identity}
Counting odd integers up to \(x\) yields the identity
\[
\pi(x) = \Big\lfloor \tfrac{x}{2} \Big\rfloor - \sum_{n\le \frac{x-1}{2}} (1-T[n]) + O(1),
\]
with boundary correction at \(x=2\). This is immediate from counting odds and subtracting composites.

\section{Proposed parametric form and verification plan}\label{sec:param}
We consider the bivariate integer form
\[
F(m,n) = 4 + 3m + 3n + 2mn, \quad m,n\in\mathbb{N}_0.
\]
\begin{claim}[Parametric coverage of odd composites]
Every odd composite \(k\ge k_0\) can be represented as \(k = F(m,n)\) for some \(m,n\in\mathbb{N}_0\) satisfying the necessary parity constraint.
\end{claim}
This statement is investigated computationally in the accompanying code by testing, for each odd composite \(k\) up to a large bound, whether 
\(n = \frac{k-3m-4}{2m+3}\) is a nonnegative integer for some \(m\ge 0\).

\section{Spectral considerations}
Define the distribution
\[
\widehat{T}(\xi) = \sum_{n\ge1} T[n] e^{-2\pi i (2n+1)\xi}.
\]
We compare numerically the magnitude of finite truncations of this series to frequencies derived from ordinates \(\gamma_k\) of nontrivial zeros \(\tfrac{1}{2}+i\gamma_k\) of \(\zeta(s)\). The mapping between continuous frequencies and discrete transforms is made explicit in the code; empirical observations are not a proof but are reproducible.

\section{Empirical validation and reproducibility}
All experiments are contained in the repository under \texttt{python/}. We include:
\begin{enumerate}[leftmargin=*]
  \item Exact agreement tests between sieve and progression methods for \(T\) up to large \(N\) (Theorem~\ref{thm:equiv}).
  \item Cross-checks against a trusted primality oracle for spot verification.
  \item Tests of parametric coverage for \(F(m,n)\) (Section~\ref{sec:param}).
  \item Spectral probes comparing selected frequencies with ordinates of zeta zeros.
\end{enumerate}

\section{Limitations and open questions}
Spectral alignment experiments suggest structure but do not constitute a proof regarding the zero locations of \(\zeta(s)\). We delineate what is empirically observed versus what is rigorously proven.

\section*{Acknowledgments}
We thank all reviewers and tool authors of open-source libraries used in the experiments.

\bibliographystyle{plain}
\begin{thebibliography}{9}
\bibitem{IwaniecKowalski} H. Iwaniec and E. Kowalski, Analytic Number Theory, AMS Colloquium Publications.
\bibitem{Montgomery} H. L. Montgomery, Ten Lectures on the Interface Between Analytic Number Theory and Harmonic Analysis.
\end{thebibliography}

\end{document}